\documentclass[10pt]{article}
\usepackage[margin=1in]{geometry}
\usepackage{setspace}
\usepackage{multicol}

\begin{document}

\vspace{2.0cm}

\begin{center}
    {\huge Still untitled [Draft] } 
    \vspace{0.4cm} \\
    {\large Michael McGirr} 
    \vspace{0.1cm} \\
    {\large Oregon State University} 
    \vspace{0.1cm} \\
    {\large Graduate School of Electrical Engineering and Computer Science}  
    \vspace{0.4cm} \\
    {\large 2016 January}  
\end{center}

\vspace{1.0cm}

\begin{multicols}{2}

    \section*{Abstract}

    This paper presents a type system that implements an ownership-style set of 
    rules for resource management.  While it may seem at the onset to be 
    unnecessary to define a system of rules governing resource management in 
    Haskell, this specific module targets the IORef and State Monad. We will 
    show some cases where this set of rules is necessary. 
    
    This ownership system introduces a set of rules that govern how, when, and
    by whom a resource is available to be used.

\end{multicols}

\end{document}
