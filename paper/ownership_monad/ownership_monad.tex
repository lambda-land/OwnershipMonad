\documentclass[onehalf,11pt]{beavtex}
\title{The Ownership Monad}
\author{Michael McGirr}
\degree{Master of Science}
\doctype{Thesis}
\department{Electrical Engineering and Computer Science}
\depttype{School}
\depthead{Director}
\major{Computer Science}
\advisor{Eric Walkingshaw}
\submitdate{TODO submit date}
\commencementyear{TODO commencement year}
\abstract{TODO abstract statement.}
\acknowledgements{TODO}

\usepackage{algorithm}
\usepackage{algorithmic}

\usepackage{hyperref} % For links

\begin{document}
\maketitle

\mainmatter

\chapter{Introduction}
\section{Introduction}

This project report presents a library for Haskell that implements an
\textit{ownership system} for resource aware programming. This ownership system
introduces a set of rules that govern how, when, and by whom a resource within
what is called an \textit{Owned Reference} (an \texttt{ORef}) can be used.

While the restrictions this implementation imposes may seem to increase the
complexity of writing programs, the resulting guaranties that adhering to these
rules facilitates offers improvements in specific areas. Furthermore
applying this system provides an example for how monadic based systems with fewer
restrictions than Linear Types can offer many of the same abilities to reason about
resource use without becoming as unwieldly as some linear type systems. 

Adding a way to keep track of resources in a pure language like Haskell may
at the onset seem unnecessary since in a pure language, by definition, % TODO Add citation
the data making up the resources bound to variables is immutable. Because of
this there is no inherent state.  Haskell's purity therefore allows for referential
transparency where values and variables can be thought of as being
interchangeable in the sense that under any context evaluating an expression will
always lead to the same result. % TODO change this sentence
Referential transparency is a very desirable property which allows for a greater
ability to reason about the behavior and correctness of a program.

Unfortunately even in a pure language like Haskell, this property breaks down in
the context of concurrency.  In order to allow separate threads to communicate
the basic mechanisms provided by Concurrent Haskell introduce mutable state.
Adding mutable state and sharing it between threads explicitly introduces
side-effects into otherwise pure and referentially transparent computations.
Under these circumstances, Haskell's usual approach of segregating side-effects
into monadic computations does not resolve every issue can exist with shared-state
concurrency.

Linear types are often suggested as a possible solution for issues that result
from mutable state \cite{Wadler90lineartypes} and concurrency
\cite{caires2010session}.  Language level support for linear types has been
proposed for the Glasgow Haskell Compiler. \cite{LinearTypesGHC}
% This would add plain linear types, directly inspired from linear logic.
Ways to add Linear types to Haskell without language level support have also
been demonstrated using embdedded domain specific languages within a monadic
context. \cite{Paykin:2017:LM:3122955.3122965}

Linear types however can become cumbersome to work with.
Other less restrictive forms logic have been used in type systems for similar
resource tracking.
Affine type systems weaken the restrictions imposed by Linear types system.
Instead of requiring every variable to be used exactly once - as is the case
with linear types - every variable must be used at most once. The language level
Ownership typing in Rust has been directly inspired by Affine type systems. 
% TODO cite other languages

Concurrent Haskell programs can still fall prey to the same fundamental
problems that other impure languages can, namely deadlock and starvation.
% TODO add citation for shared-state concurrency is still hard section from Real
% World Haskell
This project report will demonstrate what some of these problems look like using
basic concurrency tools available in Haskell such as shared state with
MVar's and message passing with channels. In Concurrent Haskell the most basic
way to allow separate threads to communicate is with a shared mutable variable.

It will then demonstrate and explain the benefits of tracking resource usage
with a set of rules similar to affine types. The contribution that
tracking resources while they are being used and sent between threads
will be shown. % TODO rework this sentence

\section{Contributions}

While this library does not do so - by tracking resources with the ownership
system it becomes in theory possible to reason about the memory usage over the
lifetime of a program using the type system of the language. 
% refer to:
%https://groups.google.com/d/msg/idris-lang/NsxReBzk7LQ/0fSPgu-4EgAJ
This method makes it possible to do a form of automatic deterministic destruction
instead of the typical garbage collection approaches.  This paper will show
where in an example program this could occur.

% enforcement of the rules can be done at compile time.

\section{Background}

Approaching resource usage with this style of implementation is not a new
concept.
Restricting all entities to following the rules specified under a affine type
system discipline is applied under the Ownership System in the Rust
programming language. 
% TODO Explain the ownership system in Rust
% Mention improvements this offers Rust

% TODO rework Idris section
Idris, which treats Uniqueness Types as a subkind of regular Types, shows
the other way of approaching this and the benefits and trade-offs of doing so.
By allowing non-unique types to exist and be used along side Unique Types,
Idris offers a degree of flexibility with it's approach to Uniqueness
Typing that is not present with ours. % But idris is using it for something
% other than concurrency - TODO explain

% TODO refer to this:
% Using the type system to encode and explain what is happening with the underlying
% resources in a general way that can be independently verified and checked greatly
% increases our confidence and ability to know when a resource is finished being used.
% refer to Edwin Brady quote at:
% https://groups.google.com/d/msg/idris-lang/Z28F3MBBtTM/XOnUcrvsBAAJ

\chapter{The Ownership Monad}

The term \textit{Ownership System} is used to describe the system for how
resources are tracked and how they can be used once they are created.
\textit{Move Semantics} describe the outcome from using the operations provided
for the \textit{Ownership System}.

The \textit{Ownership System} and \textit{Move Semantics} this library
implements are inspired by the Ownership system in Rust as well as
Uniqueness types from Idris.\cite{rust_book_ownership} \cite{idris_uniqueness_types}
The \textit{Ownership System} described by this paper approximates some of the
features from the Rust language but differences between the two result from the
different language paradigms and the different use cases.

Uniqueness types in Idris, ownership in Rust, and the \textit{Ownership Monad}
make use of the idea that by tracking resource use and applying rules to how
resources are used - certain properties can be enforced.

% TODO briefly what these are here

\section{The Ownership System}

Resources are bound to a variable once they are created inside the Ownership
Monad.  These variables are the mechanism to access - or refer - to the
underlying resource.  In the library these are called ORef's -
or \textit{Owned References}.

\subsection{Reference Creation}

An \textit{Owned Reference} is created within the Ownership Monad and bound to a
resource. When the operations inside that monad are complete - the references
will no longer exist and the resources will be marked as free.
The information inside of the resource within the \textit{Owned Reference} can
only be accessed by the provided operations for operating on references within
the \textit{Ownership Monad}.  These operations will verify whether the
ownership rules are being followed.

The newly created reference \textit{owns} the resource it was given
when it was created. Resources that are put into references can only have one
owner at any given time. This reference bound to the newly initialized resource
becomes the sole owner of that resource.

\subsection{Copying a Reference}

The underlying resource owned by a reference may be copied by other references
within the scope of that ownership monad.  When this occurs the new references is
created and is then given ownership over their copy of the resource.  After a
copy operation is performed the two references will each own what are now,
essentially, two separate and different resources.

For those familiar with the terminology from the Rust programming language, the
term \textit{copy} here is not the same as a copy in Rust. Rust makes a special
distinction between making a copy of resources that are fixed in size
\footnote{Rust will also consider an assignment operation to be a copy instead
  of a move if the \textbf{Copy} trait or the \textbf{Clone} trait is
  implemented for that type of resource. \cite{rust_book_traits}
  \cite{rust_docs_clone_trait}}
and making a copy of resources which are more complex and not fixed in size.
For the latter case it is still possible to copy these kinds of resources but these
need to be cloned (using the clone function) otherwise Rust will consider these
values to have been moved. \cite{rust_book_ownership}
With this library there is only one version of a copy and it creates a new
resource identical to the original; there is no distinction given to the kinds
of resources that are being copied.

\subsection{Moving Ownership}

A resource owned by a reference can also be transferred to a new reference or
to an existing reference. After this operation is performed it will no longer
be possible to refer to the underlying resource through the old reference. This
operation removes the old reference from the scope of the ownership monad it
previously existed in and the new reference is now the sole owner of the
resource.

Move operations provide a way for a references to interact with
other references and provide a building block for larger more complex
abstractions that will be discussed later on.

There is a key difference between moving a resource from an existing reference
and copying it to a new one.  Functionally a resource that is copied is cloned
and duplicated; doing this doubles the space and creates a new resource.
A moved resource by comparison doesn't change - instead what is altered is the
record of who owns that resource.  Neither operation, moving and copying,
creates a situation where more than one reference owns a resource.

\subsection{Immutably Borrowing a Resource}

\textit{Borrowing} a resource is the operation that allows a function to have
access to be able to use the contents of an owned reference.
A resource can be used within the confines of the ownership monad by its owner
and a function that will be required to return the resource to the context of
the ownership monad.

This operation is similar to passing a value to a function as an immutable
borrow in the Rust language.  To give some background on what this means:
depending on the type signature a function in Rust will either copy the value
it is passed, take ownership of the value, or it will borrow the value - in
which case ownership of the value is automatically returned when the function
has finished execution.\cite{rust_book_ownership}
A function in Rust that takes a borrowed value as an argument is - in a way -
syntactic sugar over that function first taking ownership of the value and then
returning ownership over the value by placing it within the expression that is
returned.
Instead of having to do these steps explicitly - a value can be passed to a
function as a borrowed value.  When a value is borrowed, the function will take
a reference to that the value from the original owner and eventually the
ownership of the resource will be handed back when the function returns.
The Rust compiler which will track the borrows (with the borrow checker).
% TODO add citation

Borrows in Rust come in two flavors - we can either lend a resource to many
borrowers as long as the borrowers never mutate the underlying resource - or we
can lend it to a single borrower that will be able to mutate the
resource.\cite{rust_book_borrowing}
It should be clear why giving multiple variables mutable access to
the same resource could create data races - which is why mutable borrows to
multiple variables (or functions) are not allowed.

This library takes a slightly different approach: instead of letting variables
borrow a resource, a borrow operation instead lends the resource to a function
which temporarily borrows the resource in order to use it.  While the resource
is being borrowed it is prevented from being written to. The function remains
inside of the context of the Ownership Monad while it executes.

Much like the Rust language will not allow for a mutable resource to be lent to
multiple borrowers - neither will the borrow operation on a \texttt{ORef}.
The \texttt{ORef} will ensure that the resource is not mutated or written to while is it
lent out to the borrowing function. 
Because each borrow operation occurs within the context of the Ownership Monad
the resource usage can be tracked and this property can be ensured. The
reference that owns the resource will track the resource while the function
executes.

The borrow operation also ensures that the original \texttt{ORef} is not able to go away
before the function borrowing it is complete - this will make sure that the
function is not referring to an \texttt{ORef} that no longer exists.  The \texttt{ORef} that is
being borrowed by the function is not able to be moved (or otherwise dropped)
before the function that the resource has been lent to is complete. This is
enforced by the \textit{move} and \textit{drop} which individually check that
a \texttt{ORef} does not have any borrowers before they operate on the \texttt{ORef}.

This library also allows multiple functions to simultaneously perform
\textit{borrow} operations on an existing \texttt{ORef}. This is equivalent to a variable
being borrowed by more than one immutable borrower in Rust. To do this a
borrower count is maintained by each \texttt{ORef} and the \textit{writable} flag is not
reset until this count is zero.

In this library the borrow operation can only read the value and will not allow
the resource to be mutated by the function.  In order to be able to borrow and
mutate the reference a different operation is required.

\subsection{Mutably Borrowing a Resource}

This library also allows a function borrowing a resource to be able to
write to (or mutate) the resource in the original \texttt{ORef}.  In Rust this would be
equivalent to having a single mutable borrower.  The \textit{mutable borrow}
operation in this library permits the function borrowing that resource to read
and set the value in the original \texttt{ORef}.

To allow a mutable borrow it is necessary to know if other functions are
borrowing the resource.  Each \texttt{ORef} tracks if it has borrowers by keeping a
ledger indicating if it is able to be read from or written to and how many
living borrowers it has.  When a resource has one or more immutable borrowers it
is no longer able to be written to - but it can still be read from.
The reason why \texttt{ORef}'s have two flags - one for read and one for write - is that
if a resource can be both read from and written to then we know it doesn't have
any borrowers; if it cannot be read from or written to we know it was either
dropped or it currently already has a mutable borrower.
In order to allow one borrower to be able to mutate the resource it is required
that it is the only borrower of the resource at that time. % TODO rework
A mutable borrow is prevented from happening is the \texttt{ORef} is not writable.

% TODO Mention the Idris approach of borrows

\subsection{Writing to a Resource}

A resource can be changed by its owner as long as it does not have any borrowers.
The value within the resource can be updated and changed through the reference
that owns the resource. This operation can be performed safely because the usage
of the underlying resource is tracked by the ownership monad.

\subsection{Dropping an Owned Reference}

The \textit{drop} operation will remove an \textit{Owned Reference} from the
\textit{Ownership Monad} it previously existing in. This will destroy the
resource from the point of view of that Owership context. Any further operations
will be prevented from occuring using the dropped reference. In order to be
dropped a resource must not have any borrowers, it must exist in that context,
and it must be readable and writable. If not the drop operation will not be
allowed to occur.


\section{Move Semantics}

\subsection{Behavior}

TODO

% With the Ownership system enforcing the rules dictating how a resource can be
% used

\section{ORef's}

TODO ORef's example section


\chapter{Owned Channels}

\section{Introduction}

\textit{Owned Channels} expand on the concept of using channels between threads
to write to and read from a shared location.   As with traditional Channels,
this location can be used by concurrent threads in order to share resources and
to communicate.

% Shared state brings with it all the dangers associated with it. REWORK
% As our example shows - issues can crop up when resources are shared between
% threads. % TODO add example

\textit{Owned Channels} operate using the idea that instead of sending just the
resource across a channel - send the ownership of the resource as well.
The key idea behind \textit{Owned Channels} is for the thread sending a
resource across a channel to relinquish ownership over the resource.
A thread reading a resource from the channel automatically gains ownership over
the resource it consumes from the channel.

Once a thread has sent a resource over the channel - all operations in the thread
will need to be prevented from using that resource.
If the thread later needed to use the resource again it would have to read the
resource from a channel and gain back ownership over the resource.

Traditionally the variable that was written to a channel would still exist in the
scope of the code in the thread that originally wrote the resource to a channel.
As a result it would be perfectly legal to write code that later referred to the
resource through an existing variable binding in the original thread.
% TODO refer to a code example for this
In order to enforce which thread owns which resource, there would need to be
some way to track not just what a resource is but also what variable (and what
thread) owns a resource.

\subsection{Using Owned References to Construct Owned Channels}

As discussed earlier, \textit{Owned References} provide a fundamental set of
rules generalizing resource ownership and how resources may be used within that
context.  Within the \textit{Ownership Monad} it is possible to use the
resources in \textit{Owned References} safely knowing that any violations of
the ownership rules will be caught and prevented.  For that reason
\textit{Owned References}  provide an useful building block for constructing
larger abstractions that are concerned with tracking resource ownership.

The key to allowing multiple (potentially mutable) concurrent operations to occur
on a shared resource is to make sure that they will not occur simultaneously.
% TODO explain that this is not a new idea and refer to existing mutex locks and
% so on?
As the chapter introduction alluded - one solution to this problem is to
create a system for shared access to resources that tracks both who owns the
resource in addition to what the resource is.
Using \textit{Owned References} as a building block it is quite easy to build
such a system. Additionally it is possible to do so on top of the existing
Channel interface and the concurrency abstractions which Channels provide in
Haskell.


\section{Owned Channel Operations}

A major idea that \textit{Owned Channels} take advantage of is that ownership
of resources - once granted - can be tracked and used in isolation. For that reason
each thread can exist inside its own \textit{Ownership Monad} bubble and
remain isolated from the state of resources which exist in other threads.
% TODO add diagram of Ownership Monad "bubbles"
The act of giving up ownership - on write operations - is enough information to
facilitate the transfer of resource ownership between threads.
% TODO add example to back this up
Beyond keeping track of this inter-thread ownership information - which is
facilitated by the \textit{Owned Channel} operations - each thread will be able
to govern its own resources.  This prevents any accidental shared ownership of
a resource from occurring and does so without having to resort to using an
additional form of communication between the threads or a resource scheduler.
This saves adding any additional overhead.



\subsection{Writing to an Owned Channel}

Writing to an \textit{Owned Channel} takes the contents of an \texttt{ORef} in
one thread and writes it to the Channel - the shared state bewteen the threads.
From the view-point of the thread that wrote to the channel - this operation
consumes the \texttt{ORef} and the thread loses the ability to further use the
\texttt{ORef} in later operations.

This operation can conceptually be thought of as the combination of two
\texttt{ORef} operations - an borrow followed by a drop operation -
% TODO should we distinguish what kind of borrow?
although these \texttt{ORef} operations are hidden from the user of the
\texttt{OChan} library.

To write an \texttt{ORef} to a Channel, the \textit{Owned Channel} operation
needs to use the value that the \textit{Owned Reference} refers to.  This means
that before any further operations can occur the \texttt{ORef} must not have any
borrowers and it must exist within the context of that \textit{Ownership Monad}.
If those conditions are satisfied then the first of the two \texttt{ORef}
operations can occur.

In order to write the resource inside the \texttt{ORef} to a traditional Channel
- inside the \textit{Owned Channel} - the function writing the resource to the
Channel needs to borrow the resource.
This might seem odd since it would appear that the borrower function does not
intend to ever return the resource.  This is accurate, the resource that was
written to the channel will not be returned; under the rules for an immutable
borrow this kind of use is actually allowed.  If it were not for the fact that
the borrower function acts on resources shared between threads there would not be
any issue.  Given the nature of concurrent operation between threads - doing
this step alone with the function to write to a channel - would be considered
dangerous.  It is also crucial to remember that this is still occurring inside
the larger \textit{Owned Channel} write operation and will be opaque to the user
of the library.

The second \texttt{ORef} operation is to drop the \texttt{ORef} that was just
borrowed. This immediately follows the completion of the borrow operation. The
borrow operation used the underlying resource in order to send it to the channel
and upon completion - as far as the \textit{Ownership Monad} for that thread is
concerned - the \texttt{ORef} no longer has any borrowers.
\footnote{Since the borrow just performed a dangerous multi-threaded IO operation
  it is not entirely true to say that it does not have any borrowers - the
  subsequent drop operation however makes this irrelevant.}
Because the \texttt{ORef} no longer has any borrowers it is now possible to
drop it from that monad using the \texttt{ORef} drop operation.  This will
update the resource ledger for that monad - crucially without touching the
resource itself - and prevents any further operations from using that resource
in that thread.

\subsection{Reading from an Owned Channel}

Reading from an \textit{Owned Channel} can be thought of as the reverse of
writing to an \textit{Owned Channel}.  Rather than take the contents of an
\texttt{ORef} and give away ownership of the resource inside it - by writing the
resource to a channel - we are taking ownership of a resource from the channel
and encapsulating it within a new \texttt{ORef}.

The first \texttt{ORef} operation inside a read from an \textit{Owned Channel}
is to read from the traditional channel inside the \textit{Owned Channel}. This
retrieves the resource from the channel and will prevent other threads from
reading it. This latter aspect leverages a previously existing aspect of
Haskell Channels. % TODO cite

The second operation is to place the freshly acquired resource inside a new
\texttt{ORef}.  This \texttt{ORef} will then be returned into the
\textit{Owership Monad} context by the read operation on the
\textit{Owned Channel}.  The thread can now have access to the \texttt{ORef}
and the value within it using the provided operations in the
\textit{Ownership Monad}.

% TODO add more on Reading

\subsection{Creating an Owned Channel}

Creating an \textit{Owned Channel}, comparatively, is a very simple operation.
A traditional channel is created through an \texttt{IO} operation and placed
within the \textit{Ownership Monad}.

It should be noted that there are write and read functions provided
in the \texttt{OChan} library that can operate on normal \texttt{IO} bound
channels.  It is recommended to not use these and instead use write and read
functions that only operate on \textit{Owned Channel} to safeguard against
accidentally trying to use the traditional channel write and read functions
with \texttt{liftIO}.


\section{Preventing Deadlock}
% Describe/Define Deadlock
TODO

\subsection{Explicit Resource Ownership Prevents Deadlock}
% How ownership prevents deadlock
TODO

\section{Beyond Threads}

TODO

How Ownership could be used for resources between processes using D-Bus and also
servers in distributed computing.

\chapter{Conclusion}
\section{TODO}

\bibliographystyle{plain}
\bibliography{ownership_monad}{}

% \appendix
% \chapter{Redundancy}

\end{document}
