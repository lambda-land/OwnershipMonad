\documentclass[onehalf,11pt]{beavtex}
\title{The Ownership Monad}
\author{Michael McGirr}
\degree{Master of Science}
\doctype{Thesis}
\department{Electrical Engineering and Computer Science}
\depttype{School}
\depthead{Director}
\major{Computer Science}
\advisor{Eric Walkingshaw}
\submitdate{TODO submit date}
\commencementyear{TODO commencement year}

\abstract{
  % Resource use is central to computation, from the
  % Von Neumann architecture onwards to modern computing systems.
  In Haskell, a purely functional language, data is usually immutable.
  In a simplistic view this means that existing resources remain unchanged and
  new copies are created in order to modify values in a program and effect change.
  This is a strong aspect of Haskell and purely functional programming.
   
  In almost every circumstance Haskell programmers can use immutable data and
  purely functional data structures to compose their programs.
  However, some situations exist in which explicitly mutable resources are
  required. 
  For example: when simultaneous concurrent threads in a program evaluate
  expressions, they often need to operate on a shared resource.  Because
  these threads are separate, the order in which the resource is operated
  on cannot be guaranteed.
  
  The basic mutable variables provided in the base libraries of Haskell, such
  as IORef's and MVar's can be used as a way to share a resource between
  threads.
  Further concurrency tools such as channels provide abstractions on top of
  these.
  % More sophisticated runtime-orientated tools like software transactional memory
  % further abstract away a user's ability to introduce subtle concurrency bugs
  % that are made possible with direct access to mutable variables. % I think
  Without resorting to more sophisticated runtime-orientated libraries,
  the tools to create mutable resources - such as IORef's and MVar's - do not provide
  a way to track resource use in concurrent programs other than in an ad hoc fashion.
  
  This project report introduces the Ownership Monad, a monad for tracking
  resource use. This project draws upon ownership typing in the Rust programming
  language as well as uniqueness typing in the Idris programming language.
  Resources placed within owned references, variable references in the Ownership
  monad, are tracked.
  While these references are internally mutable, strict rules are enforced for
  how they can be used.
  These rules are derived from Affine logic and provide the basis for how
  correct resource use can be reasoned about in the ownership monad.
  Owned references within the ownerhip monad are then used construct
  Owned Channels,
  an abstraction which sends ownership over an underlying resource to
  a thread in addition to the resource itself.
}

\acknowledgements{
  TODO
}

\usepackage{algorithm}
\usepackage{algorithmic}
\usepackage{hyperref} % For links
\usepackage{listings} % For code examples

% set listings env settings
\lstset{basicstyle=\ttfamily, frame=single, language=Haskell, numbers=left, showstringspaces=false }
% \lstset{ language=Haskell, numbers=left}


\begin{document}
\maketitle

\mainmatter

\chapter{Introduction}

This project report presents a
library\footnote{
  The library presented in this project report is available at:
  \href{https://github.com/lambda-land/OwnershipMonad}{https://github.com/lambda-land/OwnershipMonad}
}
for Haskell that implements an
\textit{ownership system} for resource aware programming. This ownership system
introduces a set of rules that govern how, when, and by whom a resource within
what is called an \textit{Owned Reference} (an \texttt{ORef}) can be used.

When an owned reference is created it is bound to a resource.
The resource is placed within an entry which is then stored in the state of the
ownership monad.
At the same time this occurs, the entry becomes the referent to a uniquely
identifiable variable ID.
This variable becomes the way to refer to and access the contents of the
resource it is bound to inside the ownership monad.
Each referent can only have, at most, one variable binding at a time.

While the restrictions this implementation imposes may seem to increase the
complexity of writing programs, the resulting guarantees offer improvements in
specific areas.
Furthermore applying this system provides an example for how monadic based
approaches can offer abilities to reason about resource usage.

Adding a way to keep track of resources in a pure language like Haskell may
at the onset seem unnecessary since in a pure language, by definition,
the data making up the resources bound to variables is immutable.
% TODO Add citation
Because of this there is no inherent state.  Haskell's purity therefore allows
for referential transparency where values and variables in closed expressions
can be thought of as being interchangeable.

Referential transparency is a desirable property which allows for a greater
ability to reason about the behavior and correctness of a program.
Even in a pure language like Haskell, this property breaks down in the context
of concurrency.
Under these circumstances, Haskell's usual approach of segregating side-effects
into monadic computations does not resolve every issue that can exist with
shared-state concurrency.

\newpage

A simple example of this is the race condition that can be created when a
mutable reference is shared across a channel between two threads:

\begin{lstlisting}
import Data.IORef
import Control.Concurrent
import Data.Char (toLower, toUpper)

introChan :: IO ()
introChan = do
  ref <- newIORef "resource"
  ch <- newChan
  writeChan ch ref

  _ <- forkIO $ do
    ref' <- readChan ch
    modifyIORef ref' (map toLower)

  val <- readIORef ref
  putStrLn val

  modifyIORef ref (map toUpper)
  newVal <- readIORef ref
  putStrLn newVal
\end{lstlisting}

Here the \texttt{forkIO} function takes an expression of type \texttt{IO ()}
and executes it in a new thread.
This new thread will run concurrently with the original parent thread.

This creates a clear race condition between the two threads and the mutable
contents of the \texttt{IORef} shared between them.
% TODO explain what an IORef is?
Both threads have access to the same \texttt{IORef} and it could be unclear
which will access the resource first.  Under these conditions the issue is not
only that we have shared mutable state but that we have shared access to that
state.

This kind of race condition would be difficult to detect. 
The easy culprit to blame in the previous example would be the
mutable \texttt{IORef} shared between the two threads.
Immutable data structures simplify concurrency and are preferable to
mutable structures. However categories of problems still exist where a
mutable structure is required.

The \texttt{OChan} and \texttt{ORef} operations provided by this report's library
allow a user to detect situations, similar to the one created by the previous
example, which would result in a race condition.
A comparable program using the abstractions provided by the ownership system
would be able to detect the race condition before it occurred:

\begin{lstlisting}
import Data.ORef
import Control.Concurrent.OChan

introOChan :: IO (Either String ())
introOChan = startOwn $ do
  ref <- newORef "resource"
  let ch = newOChan
  writeOChan ch ref

  let down :: String -> Own String
      down x = return (map toLower x)

      up :: String -> Own String
      up x = return (map toUpper x)

  _ <- liftIO $ forkIO $ do
    _childResult <- startOwn $ do
      ref' <- readOChan ch
      borrowORef' ref' down
      writeOChan ch ref'
    return ()

  borrowORef' ref up
  return ()
\end{lstlisting}

Here a resource is created and placed within an \texttt{ORef}.
The parent thread proceeds to give up ownership of the \texttt{ORef} when it
writes it to the owned channel on line 5.
The child thread can read the \texttt{ORef} from the \texttt{OChan} and claim
ownership over its contents.
The child thread can then mutate the resource in a borrow operation before
writing it back to the channel. 

Each of the ownership operations used in this example will be defined
and discussed at greater length later on.
This example demonstrates a sequence of actions that result in an ownership
violation.
This violation is occurs when parent thread tries to use a reference that it
has already given up ownership over by writing the reference to a channel.

When the overall example is evaluated it will result in a \texttt{Left} value
which will indicate the kind of ownership violation which took place.
A simple remedy for the violation can then be found.
For this example the parent thread could take back ownership of the reference by
reading it from the owned channel.

\section{Contributions}

This project draws from the use of affine types in Rust for
ownership typing to define a similar method for tracking resources with Haskell.
This is combined with the approach of defining embedded languages in monadic
programming to allow for dynamic ownership checking without language level
support for affine types in Haskell.

Concurrent Haskell programs can still fall prey to some of the same fundamental
problems that other impure languages can, namely deadlock and data races.
This project report will demonstrate a motivation for adding the kind of
resource tracking that ownership typing provides by
looking at what some of these problems look like and how resource tracking acts
to mitigate these problems.
This will use the basic concurrency tools available in Haskell
such as shared state with IORef's and message passing with channels.
It will then demonstrate and explain the benefits of tracking resource usage
with a set of rules similar to affine types.
% TODO mention how it makes it easier to debug and detect these problems

\chapter{Background}

\section{Linear and Affine Types}

A linear type system enforces the rule that resources are used exactly once.
Linear types are often suggested as a possible solution to limit the issues that
result from mutable state \cite{Wadler90lineartypes} and concurrency
\cite{caires2010session}.

Language level support for linear types has been proposed for the Glasgow
Haskell Compiler. \cite{LinearTypesGHC}
% This would add plain linear types, directly inspired from linear logic.
Ways to add linear types to Haskell without language level support have also
been demonstrated using embedded domain specific languages within a monadic
context. \cite{Paykin:2017:LM:3122955.3122965}

Other less restrictive forms of logic have been used in type systems for similar
resource tracking.
Affine type systems weaken the restrictions imposed by linear type systems.
Instead of requiring every variable to be used exactly once - as is the case
with linear types - every variable must be used at most once.
% TODO rework this - add something here
The language level ownership typing in Rust has been directly inspired by affine
type systems.


\section{Uniqueness Types}

Idris, a dependently typed language, treats uniqueness types as a subkind of
regular types. Uniqueness types in Idris are also inspired by ownership types
and borrowed pointers in the Rust programming language, as well as Uniqueness
Types in the Clean programming language. % TODO add citation.

In Idris a value that has a type that is made up of a \texttt{UniqueType}
will only be able to have at most one reference to it at run-time.
% TODO add citation.
In order to make this guarantee, any value of a unique type is kept separate
from normally typed values - which have the type \texttt{Type}.

Polymorphic functions which take either unique types or regular
types can exist.
The type \texttt{Type*} is used to denote when a type which can either be a
unique type or a regular type (but not both) may be used in a polymorphic
function.

\begin{verbatim}
head : {a : Type*} -> List a -> a
\end{verbatim}

A polymorphic function that takes either unique types or regular types will also
have to obey the rules for how values that are uniquely typed can be used
on the right hand side of a function.

% TODO add example here of uniqueness type in idris ??

Idris allows for unique typed variable to be used more than once on the right
side of a function as long as they are being read and the value is not
updated.
This means that a uniquely typed variable can have its value inspected in a
function and this will not count towards that variable being used.
The variable will not be able to be mutated or used in any way which would
mutate the underlying resource.

To do this, Idris creates a new kind of type from the unique type.
This new type is called a \texttt{BorrowedType} and at
the type level exists alongside \texttt{UniqueType} and \texttt{Type}.

Idris treats a borrowed type as a different kind of type from unique types.
Using a variable of the borrowed type adds in additional
requirements and checks at the type level to ensure that the borrowed type
will not mutate while its resource is being read.
While a borrow is occurring the internals will exclude other functions or
operations from mutating the resource.

In Idris to convert a unique type into a borrow type the \texttt{Borrowed}
dependent data type is used:

\begin{verbatim}
data Borrowed : UniqueType -> BorrowedType where
     Read : {a : UniqueType} -> a -> Borrowed a

implicit
lend : {a : UniqueType} -> a -> Borrowed a
lend x = Read x
\end{verbatim}
% TODO cite the Idris docs

The \texttt{lend} function can then be invoked to create a \texttt{BorrowedType}.
Doing so will indicate that while the underlying \texttt{UniqueType} variable is
being used new references to it should not be tracked as additional references
to a unique value.  This allows the borrowed variable to be used as many times
as it needs to be instead of at most once. 

% TODO give an example of this

\section{Ownership Types in Rust}

A central feature of the Rust programming language is its use of an
ownership-based type system. The ownership model used in Rust and the
safety claims of using such a system have been formally proven and
machine-checked in Coq \cite{Jung:2017:RSF:3177123.3158154}.

The Rust language has inherent state and additionally all variables must obey the
ownership rules.
% TODO cite the Rust book ownership rules.
These rules are checked at compile time and are enforced over all variables in
a program.
Doing so allows Rust to forgo a runtime system or a garbage collector.
% TODO add citation for this

In Rust, memory allocation is handled through the ownership system.
The ownership system is a language level abstraction which makes use of
affine types for resource management.
Additionally Rust uses \textit{lifetime} analysis of resource use for fast
deallocation. Lifetimes in Rust are beyond the scope of this project report and
in some ways are more similar to a linearly typed language than an affine one.

When a resource is created and bound to a variable, such as \texttt{v1} in the
example below, that variable is the \textit{owner} of the resource.

\begin{verbatim}
let v1 = vec!["Vector", "of", "Strings"];
\end{verbatim}

Ownership may be transferred to a different variable with a new assignment.
The variable \texttt{v2} in the next example takes ownership over the
vector which belonged to \texttt{v1}. 

\begin{verbatim}
let v2 = v1;
\end{verbatim}

Any attempts to access the values in the vector by referring to them through
the old owner \texttt{v1} will result in a compile time error.
When the variable which owns the resource goes out of scope at the end of a
code block the resource will be freed.

There may be situations in which one variable needs to temporarily
use the resource owned by another variable without taking ownership over it.
To do this a borrow of that variable is created:

\begin{verbatim}
let v1 = vec!["Vector", "of", "Strings"];
let v2 = &v1;
\end{verbatim}

Ampersands are used to create references to the resources owned by other
variables.
In the last example the variable \texttt{v2} was instantiated as a borrow
of the resource owned by \texttt{v1}.

Borrows in Rust come in two flavors. We can either lend a resource to many
borrowers as long as the borrowers never mutate the underlying resource or we
can lend it to a single borrower that will be able to mutate the
resource.\cite{rust_book_borrowing}
It should be clear why giving multiple variables mutable access to
the same resource could create data races which is why mutable borrows to
multiple variables (or functions) are not allowed.

A function in Rust depending on the type signature will either copy the value
it is passed, take ownership of the value, or it will borrow the value - in
which case ownership of the value is automatically returned when the function
has finished execution.\cite{rust_book_ownership}
A function in Rust that takes a borrowed value as an argument is
syntactic sugar over that function first taking ownership of the value and then
returning ownership over the value by placing it within the expression that is
returned.
Instead of having to do these steps explicitly a value can be passed to a
function as a borrowed value.  When a value is borrowed, the function will take
a reference to that the value from the original owner and eventually the
ownership of the resource will be handed back when the function returns.
The Rust compiler will track the borrows (with the borrow checker) statically at
compile time.

\chapter{The Ownership Monad}

The term \textit{Ownership System} is used to describe the system for how
resources are tracked and how they can be used once they are created. This
system operates within the context of the ownership monad.
The ownership system this library implements is inspired by
ownership typing in Rust \cite{rust_book_ownership}
as well as uniqueness types from Idris \cite{idris_uniqueness_types}.

The ownership system described by this paper approximates some of the
features from ownership typing in the Rust language. Differences between
the two result from the different language paradigms and the different use
cases.

Uniqueness types in Idris, ownership typing in Rust, and the ownership system in
this report make use of the idea that by tracking resources and applying rules
to how resources are used, certain properties can be enforced.
% TODO Such as? Briefly what these are here
One of the primary motivations for using the ownership monad is as a
safer way to introduce mutability, when it is needed, into a Haskell program.


\section{The Ownership System}

From the perspective of the library user: owned references are abstract
data types defined by the operations which can act on references in the
ownership monad.
The semantics of how owned references exist and operate correspond to the
outcomes of using these operations.

The functions to perform operations are made available in the public facing API
module of the library for this implementation.

\begin{verbatim}
newORef    :: Typeable a => a -> Own (ORef a)
dropORef   :: Typeable a => ORef a -> Own ()
copyORef   ::               ORef a -> Own (ORef a)
moveORef   :: Typeable a => ORef a -> Own (ORef a)
writeORef  :: Typeable a => ORef a -> a -> Own ()
borrowORef :: Typeable a => ORef a -> (a -> Own b) -> Own b
readORef   :: Typeable a => ORef a -> Own a
\end{verbatim}

These functions provide the operations to create owned references,
drop a reference from the scope, copy them, move the resource from one reference
to another,  and write a value to a reference.
Additionally there are operations for borrowing and reading the resource inside
a reference.
Borrowing allows a function to temporarily use the resource without taking
ownership over it. Reading a reference which retrieves a copy of the
value inside a reference.

\subsection{Owned References}

Owned references represent variables which provide both the symbolic entity
which owns a resources as well as the mechanism to access the underlying resource.
Resources are bound to a variable when they are created inside the ownership
monad. In the library implementation these variables are called ORef's or
\textit{Owned References}.

The type of an \texttt{ORef} is a thin wrapper around a way to tag and identify
resources stored in entries.

\begin{verbatim}
newtype ORef a = ORef {getID :: ID}
\end{verbatim}

An \texttt{ORef} is a parameterized abstract data type. Because there is only
one constructor and one field associated with this datatype it is possible to
use newtype to eliminate some of the runtime overhead.

A phantom type is used to add a type variable to each \texttt{ORef}. This
ensures the type safety of the code by requiring that a type be embedded
with each \texttt{ORef}.
This prevents references of different types from being mixed by invoking a
type error at compile time.
The polymorphic \texttt{a} allows any type to be stored in an \texttt{ORef}.
This \texttt{ORef} datatype primarily serves to provide a handle on each
resource for the ownership monad to use as it enforces ownership typing rules
and tracks each resource.

When the operations inside an ownership monad are complete the references
will no longer exist and the resources associated with each one will be marked
as free.
% TODO this is actually quite significant and worth explaining more because it
% allows us to determine when a resource no longer needs to exist.
% (can be automatically destroy and freed)
% Explain that ORef's are dropped once the monad is finished evaluating - and we
% know they can be dropped then.
The information inside the resource can only be accessed by the provided
operations for references within the ownership monad.
The operations which act on owned references will verify whether the ownership
rules are being followed and detect violations.


\subsection{Reference Creation}

References need to be introduced into the ownership monad in order to be used.
To do this a library user would create a new reference and bind it with a
resource.
Both the instantiation of the new reference as well as the resource binding
occur in the same step.
A simple example of this is the creation of a reference \texttt{x} which is
bound to the resource $\left[1,2,3\right]$.
In the following code the owned reference \texttt{x} is created.
The reference can now be used by other operations within the monad in which
it resides.

\begin{verbatim}
x <- newORef [1,2,3]
\end{verbatim}

The \texttt{newORef} function is part of the user facing API of the library
and. As the example has shown, this function creates a new owned reference
within the ownership monad and places the value provided to the function
inside the reference. This value is internally stored within an \texttt{IORef}
in the \texttt{Entry} datatype.

\begin{verbatim}
newORef :: Typeable a => a -> Own (ORef a)
newORef a = do
    (new, store) <- get
    thrId <- liftIO $ myThreadId
    v <- liftIO $ newIORef a
    let entry = (Entry Writable thrId (Just v))
    put (new + 1, insert new entry store)
    return (ORef new)
\end{verbatim}

The current state of the monad is needed in order to produce the next state
which will include the new owned reference.
The new owned reference is a mapping of an ID and the entry. The entry
contains the value being stored as well as the state of the owned reference.
Performing \texttt{get} will return the ID to use next as well as the current
mapping of owned reference ID's and their entries.

The ID of the thread which created the owned reference must be stored.
This is done in order to prevent child threads from using the owned
references that were potentially inherited through the name-spacing scope of
their parent threads.  The thread ID field of the entry is set when the owned
reference is initially created. This requires an \texttt{IO} operation to be
performed in order to get the current thread ID of the thread creating the
new \texttt{ORef}.

The entry that will be inserted into the new mapping will have a \texttt{Flag}
which is set to \texttt{Writable}.
The new store of entries, as well as the incremented ID, are put back into the
state.
The final step is to return the new \texttt{ORef} so that it can be used by other
ownership operations.
% Resources that are put into references can only have one owner at any given
% time. The reference bound to the newly initialized resource becomes the sole
% owner of that resource. 

\subsection{Dropping an Owned Reference}

Owned references can be removed from the context of the ownership monad.
A user may want to explicitly remove a reference that is no longer needed but
the ability to destroy an owned reference is also needed by other operations.

A user can destroy an owned reference using the drop operation on the owned
reference. For a reference \texttt{x} that already exists in the same context
as the drop operation this can be done as follows:

\begin{verbatim}
dropORef x
\end{verbatim}

Here the \textit{drop} operation will explicitly remove an owned reference from
the ownership monad it previously existed in.
This will destroy the resource from the point of view of the other operations
in that ownership context.

\begin{verbatim}
dropORef :: Typeable a => ORef a -> Own ()
dropORef oref = do
    ok <- checkORef oref
    case ok of
      False -> lift $
        left "Error during drop operation.\
             \ Make sure ORef intended to be dropped is writable\
             \ and within this thread."
      True -> do
        setORefLocked oref
        setValueEmpty oref
\end{verbatim}

Any further operations that try to use the dropped reference will be prevented
from occurring:

\begin{lstlisting}
startOwn $ do
  ref <- newORef "hello"
  dropORef ref
  _copy <- copyORef ref
  return ()
\end{lstlisting}

The copy operation in this example expression will not create a copy of a
dropeed reference.
Instead, because of the earlier drop operation, the copy operation will cause
this expression to evaluate to a \texttt{Left}
value: \texttt{Left "Error during copy operation"}. 

For a drop operation to occur the reference must be in the same thread as the
drop operation and the owned reference must have a \texttt{Writable} flag.
In the previous example the reference was writable prior to being dropped.
These two conditions are internally checked by the \texttt{checkORef} helper
function:

\begin{verbatim}
checkORef :: ORef a -> Own Bool
checkORef oref  = do
  entry@(Entry _f thrId _v) <- getEntry oref
  liftIO $ do
    threadId <- myThreadId
    return $ (threadId == thrId) && writable entry
\end{verbatim}

If the ownership system rules are not being violated then the ORef will be
dropped by having it's flag set to \texttt{Locked} and the value in the
entry set to \texttt{Nothing}.
The entry is not deleted from the internal mapping of
ID's to entries. Doing so allows the ownership system to determine if an ORef
has been dropped or if it never existed.

The resource space inside the entry of a dropped reference can be freed since
the Haskell runtime will garbage collect the previous value that has now been
set to the \texttt{Nothing} case.
The ability to drop an owned reference from a context will be used later
within larger abstractions.


\subsection{Copying a Reference}

A user may want to duplicate a reference in order to have two references
that at one point were copies of each other.
This would be trivial to do with regular variables but the ownership rules
which do not allow more than one binding to a resource complicate the task for
owned references.

A simple assignment of a new reference to the old reference would mean that
the two references were now referring to the same underlying resource. While
it's possible to make a copy of a variable this way we would not be able to
make any of the claims of resource use or safety.

To create a new owned reference that is a copy of an existing reference
there is the copy operation. For an owned reference \texttt{x} we can make
a copy \texttt{y} using \texttt{copyORef} as follows:

\begin{verbatim}
y <- copyORef x
\end{verbatim}

This introduces the new reference \texttt{y} into the monadic context. Initially
\texttt{y} is a copy of \texttt{x} but since the references are now separate
entities they are able to diverge in terms of what their internal
values are following the completion of the copy operation.


The copy operation allows the underlying resource owned by a reference to be
copied by other references if they are within the scope of the same ownership
monad.
When this occurs a new reference is created and then given ownership over a
copy of the resource.
After a copy operation is performed the two references will each own what are
now two separate and different resources.

This property can be observed when a reference is copied (creating another
reference) and then the original reference is mutated:

\begin{lstlisting}
startOwn $ do
  x <- newORef (1 :: Int)
  y <- copyORef x
  let f :: Int -> Own Int
      f i = return (i+1)
  borrowORef' x f
  xContents <- readORef x
  yContents <- readORef y
  return (xContents, yContents)
\end{lstlisting}

This expression will evaluate to \texttt{Right (2,1)}.
After the reference \texttt{y} was created by making a copy of
reference \texttt{x}, the contents of reference \texttt{x}
was mutated in a borrow operation (borrow operations will be explained in depth
later in the report.)
Because of this mutable action the contents inside the copy and the original
reference diverged.

\begin{verbatim}
copyORef :: ORef a -> Own (ORef a)
copyORef oref = do
    (new, store) <- get
    entry <- getEntry oref
    ok <- inThreadAndReadable oref
    case ok of
      False -> lift $ left "Error during copy operation"
      True -> do
        let newEntry = setEntryWritable entry
        put (new + 1, insert new newEntry store)
        return (ORef new)
\end{verbatim}

A copy can be made if the underlying owned reference is at least readable and
is in the same thread as the copy operation.
To perform a thread ID check the \texttt{inThreadAndReadable} function
needs to be lifted from the \texttt{IO} monad.
The \texttt{inThreadAndReadable} function will return true if the owned reference
is at least readable and in the same thread.

\begin{verbatim}
inThreadAndReadable :: ORef a-> Own Bool
inThreadAndReadable oref = do
  entry@(Entry _f thrId _v) <- getEntry oref
  liftIO $ do
    threadId <- myThreadId
    return $ (threadId == thrId) && readable entry
\end{verbatim}

An owned reference which is set to readable signals that there may be other
operations using the resource but in a way that is immutable.
If an ORef is writable this indicates that no function is currently using the
resource in a way that may mutate the value.

If the copy operation is able to occur the new owned reference is inserted into
the internal store for that monad.  The new owned reference created from the copy
operation is set as writable. The \texttt{setEntryWritable} function
creates a duplicate entry with its flag set to writable.
This action is performed even if the original reference was only
readable.

% Move to related work?
For those familiar with the terminology from the Rust programming language, the
term \textit{copy} here is not the same as a copy in Rust. Rust makes a special
distinction between making a copy of resources that are fixed in size
\footnote{Rust will also consider an assignment operation to be a copy instead
  of a move if the \textbf{Copy} trait or the \textbf{Clone} trait is
  implemented for that type of resource. \cite{rust_book_traits}
  \cite{rust_docs_clone_trait}}
and making a copy of resources which are more complex and not fixed in size.
For the latter case it is still possible to copy these kinds of resources but these
need to be cloned (using the clone function) otherwise Rust will consider these
values to have been moved. \cite{rust_book_ownership}
With this library there is only one version of a copy and it creates a new
resource identical to the original; there is no distinction given to the kinds
of resources that are being copied.

\subsection{Moving Ownership}

It may be necessary to transfer a resource from one reference to another.
Under these circumstances the old reference would need to be dropped after
the resource has been transferred, otherwise the old reference and new
reference would be referring to the same resource.

Here the resource owned by the reference \texttt{old} is
transferred to the reference \texttt{new} and in the process the
reference \texttt{old} is dropped:

\begin{verbatim}
new <- moveORef old
\end{verbatim}

A move operation transfers a resource owned by one reference to a new reference.
An existing reference can also be overwritten by the moved contents of another
reference using \texttt{moveORef'}.
After a move operation is performed it will no longer be possible to refer to the
underlying resource through the old reference.
This operation removes the old reference from the scope of the ownership monad it
previously existed in and the new reference is now the sole owner of the resource.

\begin{verbatim}
moveORef :: Typeable a => ORef a -> Own (ORef a)
moveORef oldORef = do
    ok <- checkORef oldORef
    case ok of
      False -> lift $
        left "Error during move from an oref to a new oref\
             \ check entry failed for the existing (old) oref."
      True -> do
        new <- copyORef oldORef
        dropORef oldORef
        return new
\end{verbatim}

The implementation of \texttt{moveORef} verifies that the old ORef is writable.
The old ORef must also exist within the same thread as the ownership monad in
which the operations are being performed in.

The ownership system needs to enforce that the old ORef is
writable in order for this operation to occur because this indicates that the
resource has not been dropped and is still valid to use in this context.
The ORef must also be writable in order to ensure that another operation
is not currently using the resource in a mutable way.

If the old ORef is valid and can be used in the operation, a copy of it is made
with the \texttt{copyORef} function.
After the copy is complete during the move operation, the old ORef is dropped
using the \texttt{dropORef} function.
This removes the old ORef from that monad and prevents other operations from
referring to the resource through the old ORef.
The old ORef and its entry have been dropped and internally within the state of
the monad are set to \texttt{Nothing}.
The resource space inside the entry corresponding to the old ORef can be safely
freed.

There is a key difference between moving a resource from an existing reference
and copying it to a new one.  Functionally a resource that is copied is cloned
and duplicated; doing this doubles the space and creates a new resource.
% refer to previous example from the copy section.
A moved resource by comparison doesn't change.  Instead what is altered is the
record of who owns that resource.  Neither operation, moving and copying,
creates a situation where more than one reference owns a resource.

\begin{lstlisting}
startOwn $ do
  x <- newORef "greeting"
  y <- moveORef x
  yContents <- readORef y
  return yContents
\end{lstlisting}

This expression does not violate the ownership rules and will evaluate
to \texttt{Right ``greeting''}.
If, for example, an expression tried to use a moved resource through an
old reference binding, the expression will evaluate to a \texttt{Left} value
such as for the following code:

\begin{lstlisting}
startOwn $ do
  x <- newORef "greeting"
  y <- moveORef x
  xContents <- readORef x
  yContents <- readORef y
  return (xContents, yContents)
\end{lstlisting}

This expression violates the ownership system rules and will evaluate to
a \texttt{Left} because the first \texttt{readORef} operation used is
attempting to read the resource in \texttt{x} that has been moved to \texttt{y}.


\subsection{Writing to a Resource}

The write operation exists in order for a user to be able to write over the
current value in an owned reference with a new value. This provides a way to
safely mutate the contents of the reference while at the same time discarding
it previous contents.

For example in the following expression the contents of \texttt{ref}
will be destructively updated with the new contents. This expression does
not violate any ownership rules and will evaluate
to \texttt{Right ``Some new contents''}. 

\begin{lstlisting}
startOwn $ do
  ref <- newORef "Original contents"
  writeORef ref "Some new contents"
  readORef ref
\end{lstlisting}

Write operations will follow the ownership rules. Consequently the operation
will not allow a resource to be updated if the reference no longer exists.
If the reference in the previous example was dropped prior to the
write operation the expression will evaluate to a \texttt{Left} value.
The ownership monad will prevent the write operation from occuring because
a dropped entry will no longer be writable.

\begin{lstlisting}
startOwn $ do
  ref <- newORef "Original contents"
  dropORef ref
  writeORef ref "Some new contents"
  readORef ref
\end{lstlisting}

The value within the resource can be updated and changed through the reference
that owns the resource.
This is to say that the underlying resource can be changed by referring to its
owner as long as it in scope and writable.
This operation can be performed safely because the usage
of the underlying resource is tracked by the ownership monad.

\begin{verbatim}
writeORef :: Typeable a => ORef a -> a -> Own ()
writeORef oref a = do
    ok <- checkORef oref
    case ok of
      False -> lift $
        left "Error during write operation. Checking if the entry\
             \ could be written to or if it was in the same thread\
             \ returned False."
      True -> setValue oref a
\end{verbatim}

The reference that is being changed cannot have any functions using the
\texttt{ORef} and it must be in the same thread as the write operation.
When the operation is performed the existing resource inside the \texttt{ORef}
is over written by the new value.
This naturally limits the user to overwriting the referrence with a new value
without any regards to what the previous value was inside the reference.
In order to alter the contents of a reference with a function based on what
the contents of the reference presently is, the user must use a
borrow operation.

\subsection{Borrowing a Resource}

Borrowing a resource is a library operation which allows a function to be able
to use the contents of an owned reference and have protect access to
the current value of the reference.
The function that is granted access to the resource will be required to remain
in the context of the ownership monad.

As a result, this operation can be used to lend out the contents of a reference
temporarily to be able to inspect its current contents in a print statement.
While the resource is lent out other operations will be prevented from using the
reference.

\begin{lstlisting}
startOwn $ do
  x <- newORef "Hello from inside the reference"
  let f :: String -> Own ()
      f a = liftIO $ 
             putStrLn $ "The contents of the ORef is: " ++ a
  _ <- borrowORef x f
  return ()
\end{lstlisting}

The previous expression will evaluate to \texttt{Right ()} and the contents
of the reference will be printed to standard output.
This would also be the outcome in a situation in which a long running function
borrowed the resource.

\begin{lstlisting}
startOwn $ do
  x <- newORef "Hello from inside the reference"
  let f :: String -> Own ()
      f a = liftIO $ do
        putStrLn $ "The contents of the ORef is: " ++ a
        threadDelay 1000000   -- delay for 1 second
  _ <- borrowORef x f
  contents <- readORef x
  return contents
\end{lstlisting}

Because of the monadic sequencing of operations in a single thread,
the outcome of a long acting function borrowing a function will be the same
as that of a short acting function.
The previous expression will print the contents of the referrence to
standard output, delay for some time, and will then evalutate to
a \texttt{Right} value.
For single threaded operations the ownership rules will not be violated if
a long running function borrows the resource inside a reference.

The borrow operation is similar to passing a value to a function as a mutable
borrow in the Rust language.
As we saw with the example this library takes a slightly different approach:
instead of letting variables borrow a resource, a borrow operation instead lends
the resource to a function which temporarily borrows the resource in order to
use it and potentially mutate it in place.

\begin{verbatim}
borrowORef :: Typeable a => ORef a -> (a -> Own b) -> Own b
borrowORef oref k = do
    ok <- checkORef oref
    case ok of
      False -> lift $
        left "Error during borrow operation.\
             \ The checks for if the entry was in the same thread\
             \ as the borrow operation and if the entry could be\
             \ written to returned false."
      True -> do
        setORefLocked oref
        v <- getValue oref
        b <- k v
        setORefWritable oref
        return b
\end{verbatim}

A borrow operation can occur if the owned reference is writable and
in the same thread as the operation.
While the resource is being borrowed it is prevented from being written to or
read by other operations. The \texttt{setORefLocked} function adjusts the flag
on the entry inside a owned reference to locked.

The function that is passed in a borrow can regard the value it sees as the
current state of the resource in the \texttt{ORef}. This differs from reading a
reference and making a copy of its value which will be discussed next.

The value inside a reference uses the internal library function
\texttt{getValue}. This will raise an error condition if an
attempt is made to get an empty \texttt{Nothing} value.

\begin{verbatim}
getValue :: Typeable a => ORef a -> Own a
getValue oref = do
  e <- getEntry oref
  v <- liftIO (value e)
  case v of
    Just a -> return a
    Nothing -> lift $ left "Cannot retrieve the value of an empty ORef"
\end{verbatim}

The internal \texttt{value} function is responsible for reading a value from
the \texttt{IOREf} it is stored in and reifing it to a concrete type.
This function handles the failure conditions that could come
about from casting the value inside an entry to a concrete type.

\begin{verbatim}
value :: Typeable a => Entry -> IO (Maybe a)
value (Entry _ _ (Just ioref)) = do
  v <- readIORef ioref
  case cast v of
    Just a  -> return (Just a)
    Nothing -> error "internal cast error"
value (Entry _ _ Nothing) = return Nothing
\end{verbatim}

The function that borrows the resource is of type \texttt{(a -> Own b)}. The
resource being consumed in the function remains inside of the context of the
ownership monad while the function executes. % TODO explain why this matters

Much like the Rust language will not allow for a mutable resource to be lent to
multiple borrowers - neither will the borrow operation on an \texttt{ORef}.
This will ensure that the resource is not mutated or written to while is it lent
out to the borrowing function.
Because each borrow operation occurs within the context of the ownership monad
the resource usage can be tracked and this property can be ensured. The
reference that owns the resource will track the resource while the function
executes.

The borrow operation also ensures that the original \texttt{ORef} does not
go away before the function borrowing it is complete. This will make sure that
the function is not referring to an \texttt{ORef} that no longer exists.
The \texttt{ORef} that is being borrowed by the function is not able to be moved
(or otherwise dropped) before the function that the resource has been lent to is
complete. This is enforced by the move and drop operations which individually
check that a \texttt{ORef} does not have any functions borrowing the resource.

This library does not allow multiple functions to simultaneously perform
\textit{borrow} operations on an existing \texttt{ORef}.
This restrictions means that the borrowing function is assumed to mutate the
value in the reference but it is not required to do so.

For convenience a second variant of the borrow operation is provided which
updates the contents of the borrowed reference with the result of the function. 
The type signature ensures that the function borrowing the reference will
have the same resulting type as the reference:

\begin{verbatim}
borrowORef' :: Typeable a => ORef a -> (a -> Own a) -> Own ()
\end{verbatim}


\subsection{Reading from an ORef}

A read operation is slightly different than a borrow operation. Where a borrow
operation gives a function exclusive and monitored access to a resource, a read
operation will take a snapshot of the current contents. The snapshot is returned
but no guarantees can be made about the current contents of the \texttt{ORef}
after the read operation has completed.

\begin{lstlisting}
box <- newORef True
schrodingers_cat_alive <- readORef box
\end{lstlisting}

Reading from an \texttt{ORef} will reveal what the contents of the \texttt{ORef}
was at the time the read was performed but the \texttt{ORef} will continue to
exist after the read operation. Because of this the value inside
the \texttt{ORef} may later change.

The advantage of using a read instead of a borrow is that the read operation will
remove a copy of the resource from the confines of the \texttt{ORef}.
This copied value is freed from the restrictions imposed on values
contained inside references by the ownership system.
As previous examples for the other ownership operations have shown, read
operations are useful for extracting the final value from a reference at the
completion of a series of ownerhip monad operations.

\begin{verbatim}
readORef :: Typeable a => ORef a -> Own a
readORef oref = borrowORef oref return
\end{verbatim}

\section{Internal Implementation}

The operations discussed so far which make up the public facing API are
available to the user but the implementation of these functions is hidden.
Apart from ownership monad type \texttt{Own},
% TODO also mention the functions to run the monad
the data structures and functions used in the implementation of the internal
module would also be invisible to the user of the library. The first major
piece of the internals is the ownership monad. The type of the ownership monad
internally is:

\begin{verbatim}
type Own a = StateT (ID,Store) (EitherT String IO) a
\end{verbatim}

% TODO explain EitherT as well as IO ?

State is represented using the \texttt{StateT} monad transformer.
The ownership monad needs to track the state of ownership system operations
that occur within its context but it does not need to be aware of the
operations occurring in other ownership monads.
This latter aspect will become important when discussing Owned Channels between
separate threads later on.

The state in the ownership monad is comprised of the next \texttt{ID}
to use and the current \texttt{Store}.  The \texttt{Store} maps each unique
reference \texttt{ID} to an \texttt{Entry}.

\begin{verbatim}
data Entry =
  forall v. Typeable v => Entry Flag ThreadId (Maybe (IORef v))
\end{verbatim}

The first notable parts of the \texttt{Entry} datatype are the explicitly
quantified \texttt{forall v} along with the \texttt{Typeable} typeclass
constraint on the type variable \texttt{v}.
The combination of these allows for the references in the state to be
heterogeneous.
Otherwise it would be necessary to limit the \texttt{Store} to owned
references of only one type per monadic state. % Right?
Using the \texttt{Typeable} typeclass does require that values in each entry are
\texttt{cast} in a type-cast operation.  However this is handled internally
by the library when values are retrieved from an entry.
The \texttt{cast} function reifies the generic type \texttt{v} into a real
type.
Doing so is necessary because it allows the value inside an entry to be used
as a concrete type rather than a polymorphic value.

Using the existentially quantified \texttt{forall v} also produces the
ancillary benefit that the \texttt{Entry} datatype does not need a type
variable.
As a result, the \texttt{Store} type does not need to be a parameterized abstract
data type and can instead be a simple mapping between an integer ID and an entry.

\begin{verbatim}
type Store = IntMap Entry
\end{verbatim}

Each owned reference \texttt{Entry} maintains a \texttt{Flag} to indicate the
level of access currently allowed by the ownership monad on the entry's value.
% TODO explain what each of these are.

\begin{verbatim}
data Flag = Locked
          | Readable
          | Writable
\end{verbatim}

The ID of the thread which owns the entry is also stored as part of the
\texttt{Entry} datatype. 

The value \texttt{v} in an \texttt{Entry} is maintained within the
\texttt{(Maybe (IORef v))} field of the entry datatype.
The main purpose of using the \texttt{Maybe} datatype is it allows an empty entry
to be represented.
Empty entries result when entries are dropped from one ownership monad's
context.  This allows the Glasgow Haskell Compiler to know that
the runtime memory storing the prior value in an entry can be freed.

Internally the value stored in a non-empty entry is placed in an \texttt{IORef}.
An \texttt{IORef} in Haskell is a mutable reference in the IO monad - an IO
reference. %TODO change this sentence
The ownership monad uses entries to store the values assigned to owned
references.
Even though the values in entries are mutable because of the internal \texttt{IORef}
implementation, they are encased in the ownership system types which control
access to the value.

\subsection{Running the Ownership Monad}

An ownership monad expression can be evaluated using the \texttt{startOwn}
function. This will run the expression in an initially empty ownership context,
one that does not have any existing owned references. 

\begin{verbatim}
startOwn :: Own a -> IO (Either String a)
startOwn x = runEitherT (evalStateT x (0, empty))
\end{verbatim}

The \texttt{evalOwn} function can be used to evaluate an ownership computation
with the initial context passed as an argument.

\begin{verbatim}
evalOwn :: Own a -> (ID,Store) -> IO (Either String a)
evalOwn actions startState =
  runEitherT (evalStateT actions startState)
\end{verbatim}

The \texttt{continueOwn} function can be used to evaluate a nested ownership
computation from within an existing ownership monad.  This will use the
existing monad as the context to run the nested ownership computation. 

\begin{verbatim}
continueOwn :: Own a -> Own (Either String a)
continueOwn x = do
  s <- get
  liftIO $ runEitherT (evalStateT x s)
\end{verbatim}

Functions which operate in the ownership monad, or small ownership expressions,
might exist which a user would like to run in a separate thread.
The \texttt{forkOwn} function provides a way for an ownership expression to be
evaluated in a separate thread.

\begin{verbatim}
forkOwn :: Own a -> Own ()
forkOwn innerOps = do
  _ <- liftIO $ forkIO $ do
    childResult <- startOwn innerOps
    case childResult of
      Left violation -> 
         putStrLn $ "A child thread failed with the following: " ++ violation
      Right _ -> return ()
  return ()
\end{verbatim}

The ownership expresion of type \texttt{Own a} is passed as
an argument. The \texttt{forkOwn} function evaluates the ownership operations
using a new ownership monad state. The child thread will execute the
expression it is given and will print debugging information to standard output
in the case of an ownership violation.



\section{Mitigating Deadlock}

Typically mutable resources are protected by granting exclusive access
of the resource to one thread at a time.
Other threads will have to block until the thread, which possesses access,
releases the resource.

This form of concurrency control is often referred to as a mutex, a
portmanteau of the words mutual and exclusion.
While a mutex is enough to prevent a data race, where two resources have
unfettered access to the same shared resource, without careful consideration
a mutex can easily create a deadlock situation.

Deadlock situations are common when shared resources, protected through
mutual exclusion, are nested.  MVar's (mutable variables) in Haskell can be
used to demonstrate such a situation:\footnote{ This example of lock order
  inversion deadlock adapted from the example of deadlock from Chapter 24 of
  Real World Haskell \cite{O'Sullivan:2008:RWH:1523280} }

\begin{lstlisting}
nestedResources :: MVar Int -> MVar Int -> IO ()
nestedResources outerResource innerResource = do
  modifyMVar_ outerResource $ \outer -> do
      yield
      modifyMVar_ innerResource $
                    \inner -> return (inner + 1)
      return (outer + 1)
  return ()

deadlockMVar :: IO ()
deadlockMVar = do
  resourceA <- newMVar 0
  resourceB <- newMVar 0
  forkIO $ nestedResources resourceA resourceB
  forkIO $ nestedResources resourceB resourceA
  return ()
\end{lstlisting}

The \texttt{yield} function is used here to force a context switch to the other
available runnable thread.
Lock inversion deadlock in general, and in this example
specifically, is created because in order for the function to release the outer
resource (if it possesses it) it must gain access to the inner resource.
In a multi-threaded program, one thread may gain access to one resource,
while at the same time another thread gains access to the other resource.
In such a situation both threads now need what the other thread has in order to
release their resource they already hold.

This kind of bug is difficult to find and debug because it cannot be dependably
reproduced.  The reason for this is that the situation described in this
example which creates the deadlock will not always occur.
This contributes to deadlock bugs being difficult to isolate and fix when they
do occur.

This example can be adapted to work with owned references.
When this is done, instead of spasmodically creating a deadlock condition,
it will always create an ownership violation and fail with a description of the
violation.
Instead of having to remember what order to acquire locks and where the
resources are being accessed in their program, the library user can fix the
ownership violation they introduced into their multi-threaded program:

\begin{lstlisting}
nestedORef :: ORef Int -> ORef Int -> Own ()
nestedORef outerRef innerRef = do
  borrowORef' outerRef $ \outer -> do
    liftIO $ yield
    borrowORef' innerRef $ 
                  \inner -> return (inner + 1)
    return (outer + 1)
  return ()

deadlockORef :: Own ()
deadlockORef = do
  orefA <- newORef 0
  orefB <- newORef 0
  forkOwn $ nestedORef orefA orefB
  forkOwn $ nestedORef orefB orefB
  return ()
\end{lstlisting}

This example % trivially
produces an ownership violation because owned references are only owned by the
thread which created them.
The contents of the owned references, while internally mutable, cannot be shared
across a thread as a mutable value in the same way that MVar's or IORef's can be.
This restriction is intentional; the mutability only exists within the scope of
the monad which owns the \texttt{ORef}.
% Mention that the resources in a channel are of course mutable but also
% wrapped in the ownerhip protection

The issue presented in the earlier locking access example with MVar's is rooted
in the ability to share access to mutable resources in un-intended ways.
In comparison, while owned references are mutable (as we have seen
with their internal \texttt{IORef} implementation) this is wrapped in the
ownership system rules.

\subsection{ORef's in Multi-threaded Programs}

As the previous example demonstrated owned references are only valid within the
scope of the thread that created them.  But this does not preclude them from
being used in multi-threaded programs.
In order to do that, a way to give up ownership of an \texttt{ORef} when it's
been shared between threads is needed.
The ownership system needs to prevent a situation in which a thread attempts to
use a resource that the thread has given up ownership over by sharing. 


\chapter{Owned Channels}

\section{Introduction}

\textit{Owned Channels} expand on the concept of using channels between threads
to write to and read from a shared location.   As with traditional channels,
this location can be used by concurrent threads in order to share resources and
to communicate.

Instead of sending the resource across a channel, owned channels
send ownership of the resource as well.
The key idea behind owned channels is for the thread sending a
resource across a channel to relinquish ownership over the resource.
A thread reading a resource from the channel will automatically gain ownership
over the resource it consumes from the channel.

Once a thread has sent a resource over the channel,
later operations in the thread will be prevented from using that resource.
If the thread needed to use the resource again, it would have to read the
resource from a channel and reclaim ownership over the resource.

\subsection{Motivation}

With traditional concurrent channels, the thread that originally wrote
a mutable resource to a channel would retain access to the resource. This is
because the variable bound to the resource is still valid in the scope of the
code which wrote to the channel.

As a result it would be perfectly legal to write code that later referred to the
resource through an existing variable binding in the original thread.
Channels from the collection of concurrency abstractions available in the
base libraries of Haskell do not track resource ownership as it shared
between threads.
In order to enforce which thread owns which resource, there would need to be
some way to track not just what a resource is but also what variable (and what
thread) owns a resource.

\subsection{Using Owned References to Construct Owned Channels}

As discussed earlier, the functions to operate on owned references enforce
how resources may be used within the context of the ownership monad.
It is possible to use the resources in owned references safely, knowing that any
violations of the ownership rules will be detected.
For that reason, owned references provide a useful building block for constructing
larger abstractions that are concerned with tracking resource ownership.

One such area of concern is how to enable expressions in separate
threads to access the same resource.
Multiple (potentially mutable) concurrent expressions can evaluate a shared
resource as long as they do not do so simultaneously.
It is well understood that problems arise when threads attempt to access a
shared resource without a control mechanism.
Likewise, a deadlock situation can accidentally be created
when access to shared resources is hoarded by a single thread.
As the chapter introduction alluded, one solution to this problem is to
create a system for sharing access to resources between channels which tracks
both what the resources are as well as who owns the resources.

Using owned references as a building block it is quite easy to create
such a system.  Additionally it is possible to do so on top of the existing
channel interface and the concurrency abstractions which the base libraries
in Haskell provide.

\section{Owned Channel Operations}

The major idea that owned channels take advantage of is that ownership
of resources can be tracked and used in isolation. Each ownership monad only
needs to concern itself with the references that are in the scope of that monad.

For that reason multiple threads can exist and each can have their own
ownership monad without sacrificing the ability to reason about resource use.
The resources available to each of these threads are covered by that thread's
ownership monad.
The act of giving up ownership (on write operations) is enough information to
facilitate the transfer of resource ownership between threads.
% TODO add example to back this up

Owned channels are the abstraction built on top of the ownership monad which
handle the inter-thread ownership information. The operations are intended
to be similar to the channel operations from \texttt{Control.Concurrent.Chan}
in the base libraries of Haskell. There are operations for creating an owned
channel, writing to the owned channel and reading from an owned channel:

\begin{verbatim}
type OChan a = Own (Chan a)

newOChan   :: OChan a
writeOChan :: Typeable a => OChan a -> ORef a -> Own ()
readOChan  :: Typeable a => OChan a -> Own (ORef a)
\end{verbatim}

Owned channel operations detect ownership violations and prevent the
accidental shared ownership of a resource.
This allows each thread to be able to acquire or give up ownership of resources
in a safe manner. The read and write operations will only allow the user to
read or write references to the channel. Read and write operations are also
restricted to the context of the ownership monad.

\subsection{Writing to an Owned Channel}

Writing to an owned channel takes the contents of an \texttt{ORef} in
one thread and writes it to the channel - the shared state between the threads.
From the view-point of the thread that wrote to the channel, this operation
consumes the \texttt{ORef} and the thread loses the ability to further use the
\texttt{ORef} in later operations.

\begin{verbatim}
writeOChan :: Typeable a => OChan a -> ORef a -> Own ()
writeOChan ch oref = ch >>= (\x -> writeOChan x oref)
\end{verbatim}

The \texttt{writeOChan} function takes an owned channel and writes the
resource to it. Because the owned reference only exists within the
ownership monad it is not necessary for resource safety to use
an \texttt{OChan} and a regular \texttt{Chan} type can be substituted
in it's place if used in combination with the \texttt{writeOChan'} function.
Using an \texttt{OChan} with \texttt{writeOChan} saves the user
from having to lift the regular channel out of the IO monad within their
ownership monad expression and is provided for convenience.

This write operation is composed from the combination of two
\texttt{ORef} operations, a borrow followed by a drop operation, 
although these \texttt{ORef} operations are hidden from the user of the
\texttt{OChan} library.

\begin{verbatim}
writeOChan' :: Typeable a => Chan a -> ORef a -> Own ()
writeOChan' ch oref = do
  borrowORef oref (\v -> liftIO $ writeChan ch v)
  dropORef oref
\end{verbatim}

To write an \texttt{ORef} to a channel, the owned channel operation
needs to use the value that the owned reference refers to.  This means
that before any further operations can occur the \texttt{ORef} must not have any
functions currently using its contents. The owned reference must exist within
the context of the ownership monad as well.
If those conditions are satisfied then the first of the two \texttt{ORef}
operations can occur.

In order to write the resource inside the \texttt{ORef} to a traditional
channel, within the owned channel operation, the function writing the resource
to the channel needs to borrow the resource.
This might seem odd since it would appear that the function borrowing access
to the resource intends to send the resource to another thread.
This is accurate, the resource that was written to the channel will not be
returned and under the rules for a borrow this kind of use is allowed.
This is similar to a resource being duplicated using the copy operation, the
only difference is that the function which is borrowing access 
performs an \texttt{IO} operation to send a copy of the resource to the
channel.

This step stands out because the function borrowing the resource is sharing
it with other threads and with different ownership monads.
However, this step occurs inside the larger owned channel write operation and will
be opaque to the user of the library.

When the resource has been written to the channel it will still exist in the
original thread.
This is resolved in the next \texttt{ORef} operation which is to drop
the \texttt{ORef} that was just borrowed.
This immediately follows the completion of the borrow operation which used the
underlying resource in order to send it to the channel.
Upon completion, as far as the ownership monad for that thread is
concerned, the \texttt{ORef} has been consumed.
\footnote{The value sent across the channel will not be in an IORef.
  The borrow operation allows a function act on the contents of the IORef in
  an ORef entry but not the IORef itself.}

Because the \texttt{ORef} no longer has any functions using it, it is now
possible to drop it from that monad using the \texttt{ORef} drop operation.
This will update the resource ledger for that monad - crucially without
touching the resource itself.
This prevents any further operations from using the resource
in that thread.

\newpage

\begin{lstlisting}
singleThreadedWrite :: Own ()
singleThreadedWrite = do
  ref <- newORef ""
  writeORef ref "Quark"
  let ch = newOChan
  writeOChan ch ref
  writeORef ref "Odo"
  -- ^^ writing to a ref that's no longer owned
\end{lstlisting}

This example will evaluate to a \texttt{Left} value which identifies a
violation caused by a \texttt{writeORef} operation in this monad.
After the \texttt{ORef}, named \texttt{ref}, is written to the owned channel on
line 6 it no longer exists within this ownership environment.
For that reason when the expression tries to use \texttt{ref} on line 7, an
ownership violation is detected.

A write operation should totally consume the resource.
Even though a read operation seems suitable for the first internal step in a
write operation, the borrow operation on owned references is used instead.
This is done in order to ensure that the actual value in the owned reference is
being used and sent across the channel.
The read operation would report what the value was at the time of reading it but
other functions could then mutate it in quick succession after the read is
complete. % but after a borrow this could happen as well ...

\subsection{Reading from an Owned Channel}

Reading from an owned channel can be thought of as the reverse of writing to an
owned channel.
Rather than sending the contents of an \texttt{ORef} to another thread and giving 
up ownership of the resource,
we are instead taking ownership of a resource from the channel and encapsulating
it within a new \texttt{ORef}.

\begin{verbatim}
readOChan :: Typeable a => OChan a -> Own (ORef a)
readOChan oc = oc >>= readOChan
\end{verbatim}

A normal channel can be used with the \texttt{readOChan'} function because
the reference which is read from the channel will remain within the context
of the ownership monad. 

\begin{verbatim}
readOChan' :: Typeable a => Chan a -> Own (ORef a)
readOChan' ch = do
  v <- liftIO $ readChan ch
  newORef v
\end{verbatim}

Within the read operation a resource is first read from the traditional
channel.
This retrieves and removes the resource from the channel - which prevents other
threads from then reading that resource.

The second operation is to place the freshly acquired resource inside a new
\texttt{ORef}.  This \texttt{ORef} will then be returned into the
final ownership monad context.
The thread which performed the read operation will now have access to the
\texttt{ORef} and will be able to manipulate the value inside it using the
provided operations for owned references. \\

\begin{lstlisting}
chanTest :: Own ()
chanTest = do
  ch <- newOChan
  ref <- newORef ""
  writeORef ref "Quark"
  writeOChan' ch ref
  _childThrID <- forkOwn $ do
    -- in the child thread
    childRef <- readOChan' ch
    val <- readORef childRef
    liftIO $ putStrLn $ "Child thread received: " ++ val
    liftIO $ do 
      yield
      threadDelay 3000000  -- 3 seconds
    writeOChan' ch childRef
    return ()
  -- back in the parent thread
  newParentRef <- readOChan' ch
  writeORef newParentRef "Odo"
  return ()
\end{lstlisting}

The previous example shows a set of operations that will not create an ownership
violation. In this example an owned channel is created in the parent thread and
an owned reference is written to the channel.
This will remove that reference from the parent environment.
The thread is then forked with \texttt{forkOwn} which will evaluate the
ownership expression passed to it in a child thread.
The child thread reads the owned reference from the channel and prints it.
Even though the child thread then delays for three seconds (line 14) before
writing the reference back to the owned channel, there will not be an ownership
violation in the parent thread.
The parent thread regains ownership over the reference by reading it from the
owned channel an is able to operate on it once again.

\subsection{Creating an Owned Channel}

Creating an owned channel, comparatively, is a very simple operation.
A traditional channel is created through an \texttt{IO} operation and placed
within the ownership monad.

\begin{verbatim}
newOChan :: OChan a
newOChan = do
  ch <- liftIO newChan
  return ch
\end{verbatim}

It was noted earlier that there are write and read functions provided
in the \texttt{OChan} library that can operate on normal \texttt{IO} bound
channels.  Using a regular channel in the ownership monad,
with the write and read operation for owned channels,
provides the same ability to reason about resource use and detect violations.



\chapter{Conclusion}

This project report has demonstrated a monadic library-based approach for
tracking resources.
Resource management done in this way allows a user to detect when a program
tries to a use a resource in a referent with more than one variable binding.

Other languages have approached resource management with similar
solutions: notably language level support for ownership typing in Rust as well
as uniqueness types in Idris which make use of Idris' dependent type system.
This report has demonstrated how a similar approach to tracking resource use can be
implemented in Haskell as a library.

The library approach shows how a similar ability for reasoning about resource use
can be attained without abstractions which require language level support for
affine or linear types.
While systems based on these approaches may provide more features for resource
tracking, the resource tracking functionality defined in the ownership
monad leverages existing language features and libraries in Haskell.

This report has also shown that the operations presented with the ownership monad
can be used to construct larger abstractions which concern themselves with
resource use.

\bibliographystyle{plain}
\bibliography{ownership_monad}{}

% \appendix
% \chapter{Redundancy}

\end{document}
